%%%%%%%%%%%%%%%%%%%%%%%%%%%%%%%%%%%%%%%
% Deedy - One Page Two Column Resume
% LaTeX Template
% Version 1.2 (16/9/2014)
%
% Original author:
% Debarghya Das (http://debarghyadas.com)
%
% Original repository:
% https://github.com/deedydas/Deedy-Resume
%
% IMPORTANT: THIS TEMPLATE NEEDS TO BE COMPILED WITH XeLaTeX
%
% This template uses several fonts not included with Windows/Linux by
% default. If you get compilation errors saying a font is missing, find the line
% on which the font is used and either change it to a font included with your
% operating system or comment the line out to use the default font.
% 
%%%%%%%%%%%%%%%%%%%%%%%%%%%%%%%%%%%%%%
% 
% TODO:
% 1. Integrate biber/bibtex for article citation under publications.
% 2. Figure out a smoother way for the document to flow onto the next page.
% 3. Add styling information for a "Projects/Hacks" section.
% 4. Add location/address information
% 5. Merge OpenFont and MacFonts as a single sty with options.
% 
%%%%%%%%%%%%%%%%%%%%%%%%%%%%%%%%%%%%%%
%
% CHANGELOG:
% v1.1:
% 1. Fixed several compilation bugs with \renewcommand
% 2. Got Open-source fonts (Windows/Linux support)
% 3. Added Last Updated
% 4. Move Title styling into .sty
% 5. Commented .sty file.
%
%%%%%%%%%%%%%%%%%%%%%%%%%%%%%%%%%%%%%%%
%
% Known Issues:
% 1. Overflows onto second page if any column's contents are more than the
% vertical limit
% 2. Hacky space on the first bullet point on the second column.
%
%%%%%%%%%%%%%%%%%%%%%%%%%%%%%%%%%%%%%%


\documentclass[]{deedy-resume-openfont}
\usepackage{fancyhdr}
 
\pagestyle{fancy}
\fancyhf{}
 
\begin{document}

%%%%%%%%%%%%%%%%%%%%%%%%%%%%%%%%%%%%%%
%
%     LAST UPDATED DATE
%
%%%%%%%%%%%%%%%%%%%%%%%%%%%%%%%%%%%%%%
% \lastupdated

%%%%%%%%%%%%%%%%%%%%%%%%%%%%%%%%%%%%%%
%
%     TITLE NAME
%
%%%%%%%%%%%%%%%%%%%%%%%%%%%%%%%%%%%%%%
\namesection{Manni Bhaskar}{Mallik}{ \urlstyle{same}\href{https://www.linkedin.com/in/manni-bhaskar-mallik/}{linkedin.com/in/manni-bhaskar-mallik/}\\
\href{mailto:mannibhaskar4@gmail.com}{mannibhaskar4@gmail.com} | \href{tel:8102087043}{+91 81020-87043}
}

%%%%%%%%%%%%%%%%%%%%%%%%%%%%%%%%%%%%%%
%
%     COLUMN ONE
%
%%%%%%%%%%%%%%%%%%%%%%%%%%%%%%%%%%%%%%

\begin{minipage}[t]{0.33\textwidth} 

%%%%%%%%%%%%%%%%%%%%%%%%%%%%%%%%%%%%%%
%     SKILLS
%%%%%%%%%%%%%%%%%%%%%%%%%%%%%%%%%%%%%%

\section{Skills}
\subsection{Languages}
Java \textbullet{} HTML \textbullet{} CSS \textbullet{} Javascript \\
\textbullet{} C  \textbullet{} Python \textbullet{} MATLAB \textbullet{} ReactJs




\subsection{Tools}
Git \textbullet{} GitHub  \textbullet{} MATLAB Online  \textbullet{} \\ Jupyter Notebook \textbullet{} JupterLab \\ \textbullet{} Apache Tomcat   \textbullet{} JDBC \textbullet{} IntelliJ IDEA \\ \textbullet{} VSCode  \textbullet{} Eclipse \textbullet{} Sublime Text
\sectionsep

\subsection{DBMS}
Oracle  
\sectionsep

\subsection{Operating Systems}
Linux \textbullet{} Windows 
\sectionsep

\subsection{Soft Skills}
 Writing \textbullet{} Time Management
\sectionsep

%%%%%%%%%%%%%%%%%%%%%%%%%%%%%%%%%%%%%%
%     LINKS
%%%%%%%%%%%%%%%%%%%%%%%%%%%%%%%%%%%%%%

\section{Links} 
LinkedIn://  \href{https://www.linkedin.com/in/manni-bhaskar-mallik/}{\bf manni-bhaskar-mallik} \\
GitHub://  \href{https://github.com/mannibhaskar4}{\bf mannibhaskar4} \\

\sectionsep

%%%%%%%%%%%%%%%%%%%%%%%%%%%%%%%%%%%%%%
%     EDUCATION
%%%%%%%%%%%%%%%%%%%%%%%%%%%%%%%%%%%%%%

\section{Education} 

\subsection{BGI - SDET}
\descript{B. Tech in Computer Science \\ and Engineering}
\location{June 2021 | West Bengal, India}
\location{ Cum. GPA: 7.5 / 10.0}
\sectionsep

%%%%%%%%%%%%%%%%%%%%%%%%%%%%%%%%%%%%%%
%     COURSEWORK
%%%%%%%%%%%%%%%%%%%%%%%%%%%%%%%%%%%%%%

\section{Coursework}

\subsection{Undergraduate}
\sectionsep
Database Management System \\
Computer Networks \\
Operating Systems \\
Object Oriented Programming \\
Data Structures \& Algorithms \\
Design \& Analysis of Algorithms \\
Digital Electronics \\
Discrete Mathematics \\
Computer Organisation \\
Computer Architecture \\

%%%%%%%%%%%%%%%%%%%%%%%%%%%%%%%%%%%%%%
%
%     COLUMN TWO
%
%%%%%%%%%%%%%%%%%%%%%%%%%%%%%%%%%%%%%%

\end{minipage} 
\hfill
\begin{minipage}[t]{0.66\textwidth} 

%%%%%%%%%%%%%%%%%%%%%%%%%%%%%%%%%%%%%%
%     EXPERIENCE
%%%%%%%%%%%%%%%%%%%%%%%%%%%%%%%%%%%%%%



%%%%%%%%%%%%%%%%%%%%%%%%%%%%%%%%%%%%%%
%     PROJECTS
%%%%%%%%%%%%%%%%%%%%%%%%%%%%%%%%%%%%%%

\section{Projects}

\runsubsection{Net-Banking Management System}\\
\location{Dec 2019 – Jan 2020 | Guided Individual Project}
\sectionsep
\begin{tightemize}
\item Developed to help customer for online banking using \textbf{J2EE}.
\item \textbf{JDBC} was used for data connection with  \textbf{Oracle}.
\item \textbf{JSP} and \textbf{Servlet} were for request.

\end{tightemize}
\sectionsep

\runsubsection{Video Library Management System}\\
\location{Mar 2019 – April 2019 | Guided Individual Project}
\begin{tightemize}
\item Used \textbf{C\#} and \textbf{.Net} for creating project.
\item Interface for the library management was designed in \textbf{Visual Studio 2013}. 
\end{tightemize}
\sectionsep

\runsubsection{React Amazon Clone}\\
\location{Aug 2020 – Sept 2020 | Guided Individual Project}
\begin{tightemize}
\item Used \textbf{React}, \textbf{CSS} and \textbf{npm APIs} for creating project.
\item This clone have basic functionalities of amazon shopping website. 
\end{tightemize}
\sectionsep


%%%%%%%%%%%%%%%%%%%%%%%%%%%%%%%%%%%%%%
%     Accomplishments
%%%%%%%%%%%%%%%%%%%%%%%%%%%%%%%%%%%%%%



%%%%%%%%%%%%%%%%%%%%%%%%%%%%%%%%%%%%%%
%     Certifications
%%%%%%%%%%%%%%%%%%%%%%%%%%%%%%%%%%%%%%

\section{Certifications} 
\sectionsep
\subsection{Problem Solving Through Programming in C}
NPTEL18CS10S3510067\\
\location{May 2018}
\sectionsep

\sectionsep
\subsection{What is Data Science?}
\href{https://www.coursera.org/verify/8JF9WKN6YBBM}{coursera.org/verify/8JF9WKN6YBBM}\\
\location{Sept 2020}
\sectionsep

\sectionsep
\subsection{Tools for Data Science}
\href{https://www.coursera.org/verify/AAQJRPAWZFLK}{coursera.org/verify/AAQJRPAWZFLK}\\
\location{Sept 2020}
\sectionsep


\sectionsep
\subsection{Data Science Methodology}
\href{https://www.coursera.org/verify/BPKDL9CTLAF9}{coursera.org/verify/BPKDL9CTLAF9}\\
\location{Sept 2020}
\sectionsep

\sectionsep
\subsection{Python for Data Science and AI}
\href{https://www.coursera.org/verify/69CAX92V3KP5}{coursera.org/verify/69CAX92V3KP5}\\
\location{Sept 2020}
\sectionsep

\end{minipage} 
\end{document}  \documentclass[]{article}
